\pdfoutput=1
\documentclass[12pt]{article}%
\usepackage[utf8]{inputenc} % pouľito v případě jiného kódování
% aktuální kódování: ISO Latin 2 (ISO 8859-2)
\usepackage{czech, dipp8}
\usepackage[usenames,dvipsnames]{color}
%\usepackage[usenames,dvipsnames,svgnames,table]{xcolor}

\begin{document}

%brava pro pozadi boxu urceneho pro prikazy
\definecolor{light-gray}{gray}{0.95}

\def\,{\penalty10000\hskip.25em}
\pagestyle{headings}

\cislovat{2}
\bakalarska

\titul{Tvorba geografického informačního systému arboreta MENDELU}
{Bc. David Ruber}{Ing. David Procházka, Ph.D.}{Brno 2014}

\newpage
Zde prostor pro zadání práce
\newpage

\podekovani{Rád bych poděkoval svému vedoucímu práce Davidu Procházkovi za ochotu, odborné
konzultace v rámci této práce.}

\prohlasenimuz{V~Brně dne \today}

\abstract{Bc. Ruber, D. Creating a geographic information system of MENDELU arboretum.
Diploma thesis. Brno: MENDELU, 2016}
{abstrakt}
\\
\\
{{\noindent{\sffamily\bfseries Keywords}}
\\
{klicova slova}

\abstrakt{Bc. Ruber, D. Tvorba geografického informačního systému arboreta MENDELU. Diplomová práce.
Brno: MENDELU, 2016}
{abstrakt}
\\
\\
{{\noindent{\sffamily\bfseries Klíčová slova}}
\\
{klicova slova}

\cislovat{3}
\obsah


\kapitola{Úvod a~cíl práce}
\sekce{Úvod}
Geografický informační systém umožňuje zobrazovaní a manipulaci s prostorovými daty. Zachycuje 
reálný svět s údaji o poloze jednotlivých částí našeho okolí. Poloha samotná nestačí, pro úplnou 
definici objektu, musejí se uchovat i ostatní jeho informace. Objekty reálného světa se navzájem 
ovlivňují, mají určitou vzdálenost mezi sebou a proto GIS slouží v mnoha lidských činnostech. Každý
člověk pravděpodobně nejméně jednou použil určitou GIS aplikaci k nalezení hledaného místa, 
navigování k danému místu nebo například k získání informací. Všechna funkcionalita funguje hlavně 
díky možnostem provádění prostorových analýz nad prostorovými daty. Data samotná nemohou reagovat 
na dotazy uživatelů. K těmto účelům se používají geodatabáze, které mohou být pouhou nádstavbou 
relační databáze. Velmi důležitou částí se tak stává návrh struktury databáze. V práci bude popsán 
návrh s odůvodněními. Jen návrh databáze nestačí ke kompletnímu návrhu informačního systému. Před 
implementací existují další fáze vývoje, kdy se například musí zjistit požadavky uživatelů, vytvořit
vzhled a umístění prvků rozhraní aplikace.

\sekce{Cíl práce}
Cílem práce je vytvoření geografického informačního systému arboreta mendelu. Základní prvky 
vycházejí z bakalářské práce, která popisuje tvorbu geodatábaze a jednoduché zobrazení dat. 
Vytvoření vyžaduje prostudování dostupných softwarových prostředků, které lze použít. Výběr 
nejlepšího řešení s ohledem na minimalizaci nákladů pro tvorbu. Celkový informační systém potřebuje 
i další prvky, mezi které patří například administrační část nebo uživatelsky přívětivé rozhraní. 
Nejdříve bude nutné prozkoumat požadavky na samotný systém, jaká funkcionalita má být implementována, 
co je nutné uchovat za informace atd. Předchozí posbírané data poslouží k tvorbě návrhu aplikace 
před samotnou implementací. Druhým důležitým cílem je vytvoření dokumentace s podrobným popisem 
reálných situací při práci se systémem. Mezi situace se zařadí manipulace s mapou, manipulace s daty 
ve smyslu přidávání, oprav, mazání.

%predstaveni dostupneho sw
\kapitola{Dostupný software}
%popis djanga
%https://www.djangoproject.com/
%http://www.djangoproject.cz/
\sekce{Django}
Django je webový framework napsaný v~jazyce Python. Byl vyvinut aby pomohl vývojářům urychlit 
vývoj aplikací od~základního konceptu až po finální dokončení. K~těmto účelům poslouží jak oficiální 
návod krok po~kroku, ve~kterém jsou popsány všechny základní kroky, tak i~volně dostupná 
dokumentace obsahující všechny specifika. Django obsahuje velké množství již kompletně připravených 
modulů, které lze využít, jako jsou například autentizace uživatelů, mapu stránek, RSS kanál a~další. 
Ověřování uživatelů je jedním z~bezpečnostních prvků, které usnadňují práci vývojářům. Mezi další
bezpečností prvky patří SQL injection\footnote{SQL injection je technika napadení databázové vrstvy 
prostřednictvím neošetřeného vstupu.}, cross-site scripting\footnote{Metoda využívající bezpečnostní 
chyby ve skriptech a~vložení vlastních skriptů.}, cross-site request forgery\footnote{Metoda 
pracující na~bázi nezamýšleného požadavku pro~vykonání určité akce.} a clickjacking\footnote{Způsob 
útoku kdy uživatel spustí akci, kterou nepředpokládal.}. Dajngo využívá architekturu "sdílené nic", 
což znamená že~každý uzel je soběstačný a~nezávislý. Přesněji žádný z~uzlů nesdílí pamět nebo 
diskové úložiště. Framework odděluje jednotlivé komponenty (datábazovou vrstvu, aplikační vrstvu 
atd.).
%popis geo djanga
%https://docs.djangoproject.com/el/1.10/ref/contrib/gis/
\podsekce{GeoDjango}
GeoDjango je samostatný modul pro Django, který obsahuje geografické rozšíření pro~webové aplikace. 
Modul usiluje o~to, aby byl, stejně jako samostatný Django framework, co nejvíce jednochý pro 
vývojáře. Obsahuje vlastní modelá pole pro OGC\footnote{Open Geospatial Consortium je mezinárodní 
standartizační organizace.} geometrii a~rastrová data. Rozšiřuje Django ORM\footnote{Objektově 
relační zobrazení je programovací technika zajišťující konverzi dat mezi relační databází 
a~objektově orientovaným pragramovacím jazykem} pro potřeby geografických požadavků a~prostorových 
dat. Python rozhraní pro geometrii, rastrové operace a~práci s~daty v různých formátech je na 
vysoké úrovni.

%popis google maps
%https://www.google.com/maps/about/
%https://developers.google.com/maps/documentation/api-picker
\sekce{Google Maps}
Nejznámější mapovou službou jsou bezpochyby Google Maps. První verze Google Maps API vzinkla 
v roce 2005. API není jen JavaScript pro koncového uživatele, zahrnuje více projektů. Nabízí 
několik různých způsobů jak službu vložit do webových stránek s hodně možnostmi přizpůsobení. 
Dostupné API služby zahrnují například Google Maps API Javascript, Google Static Maps API, Google 
Maps SDK pro iOS nebo Google Maps pro Android. Základní APi jsou zdrama, ale platí určitá omezení. 
Prvním je, že mapy musí být volně a veřejně přístupné. Druhým je omezení využití, které omezuje to, 
aby web negeneroval více než povolené denní množství map. Existuje i placené řešení API.

%popis leafletu
%http://leafletjs.com/
\sekce{Leaflet}
Leaflet je nejrozšířenější open-source knihovna v~jazyce JavaScript pro tvorbu interaktivních map. 
Obsahuje většinu potřebných mapových funkcí, jako například:
\begin{itemize}
\item podkladové mapy, WMS\footnote{Web Map Service je otevřený standart pro~sdílení geografických 
informací.}
\item vektorové vrstvy: polygony, linie atd.
\item mapové značky
\item GeoJSON
\end{itemize}
Knihovna obsahuje již naimplementované prvky pro~ovládání mapy, mezi které patří gesta prsty nebo 
dvojité kliknutí myši pro příblížení, ovládání skrze klávesnici (pohyb pomocí šipek), nebo 
připravené ovládací panely zobrazené přímo v~mapě. Leaflet je navržen tak, aby byl jednoduchý, 
výkonný a~použitelný. Jeho hlavním kladem je funkčnost skrze všechny internetové prohlížeče 
stolních počítačů i mobilních zařízení. Samotné jádro knihovny klade důraz na jednoduchost, 
výkonnost a použitelnost.

%popis openlayers
%https://openlayers.org/
\sekce{OpenLayers}
OpenLayers je stejně jako Leaflet jedna z~nejrozšířenějších open-source JavaScript knihoven pro 
dynamické mapy umístěné na webových stránkách. Byl vyvinut společností MetaCraft jako ekvivalent 
ke~Google Maps. Umožňuje zobrazovat podkladové mapy, vektorová data, mapové značky a~další. 
Podkladové mapy je možnénačíst z mnoha zdrojů, např: OSM, Bing, MapBox. Stejně tak jsou 
podporovány OGC mapové služby. OpenLayers umožňuje i~vykreslení vektorových vrstev GeoJSON, KML, 
GML a~dalších formátů.

\sekce{Srovnání Leaflet a OpenLayers}
Porovnat Leaflet s OpenLayers je v celku jednoduché. Leaflet je zaměřen na jednoduchost a proto 
jeho použití je jednodušší. Je správným řešením pro jednoduché bežné mapové úkoly jako je zobrazení 
mapových podkladů a částí, posouvání mapy nebo zoomování. Celkové API je na vyšší úrovní, což 
znamená méně kódu a je jednodušší na pochopení, není tak rozsáhlé jako u OpenLayers. Openlayers 
nabízí více funkcí ovšem za cenu většího množství kódu, nutnosti inicializace a nastavení mapy. 
API je na nižší úrovní a proto je težší na pochopení a naučení se potřeb pro použití.
Díky inicializaci je dosáhnuta vyšší kontrola nad mapou a daty. V rámci kontroly je dobře zvládnuta 
podpora editace. Leaflet nabízí méně funkcí, ale existuje mnoho plug-in\footnote{Plug-in je 
zásuvný modul, který doplňuje aplikaci.}, které chybějící prvky doplní.

%\sekce{PHP Brick\textbackslash Geo}

%popis crowdfundingu
%https://www.fundable.com/learn/resources/guides/crowdfunding-guide/what-is-crowdfunding
\kapitola{Crowdfunding}
Crowdfunding je způsob získávání pěnez na~projekt formou veřejné sbírky. Nejedná se o~tradiční 
přístup k~financování podnikatelské činnosti. Většinou je nutné provést průzkum trhu, nastavit 
svůj podnikatelský plán vytvořit prototyp a~následně produkt propagovat před investory. Zdroje 
financování jsou pak banky, investoři, nebo společnosti poskytující kapitál. Ve výsledku podobným 
přístupem je na konci omezené množství investorů. Crowdfunding tvoří přesný opak, na konci lze mít 
velké množství investorů. Pro získání peněz postačí vytvořit profil produktu na jedné z~mnoho 
webových stránek, nabídnout za příspěvek například první vyrobený kus, oficiální uvedení 
přispěvatele nebo jiné bonusy a~potom jen čekat na příspěvky, dokud není naplněn základní kapitál.
\sekce{Výhody crowdfundingu}
Existuje mnoho výhod oproti tradičním metodám, uvedené jsou jen některé z nich. Při použití 
crowdfunding platformy je získán přístup k tisícům akreditovaných investorů, kteří mohou 
komunikovat a dále předávat prezentovaný produkt. Vytvoření prezentace pomůže mapování historie 
vývoje produktu, jeho statistik z pohledu úspěšnosti a hlavně jako zpětná vazba od obyčejných 
lidí(investorů), kteří mohou vyjádřti svůj názor a tím přispět k novým vylepšením. Přístup 
získáva na~popularitě zejména díky sociálním sítím, na kterých se projekty mohou snadno šířit 
a díky tomu získávat více potencionálních přispěvatelů. Kromě sociálních sítí lze projekt šířit 
pomocí emailových newsletterů nebo vlastní webovou prezentací. 
Stejně jako klasických metod pro zahájení výroby, tak i typů crowdfundingu existuje více. Mezi 
základní patří:
\begin{itemize}
\item Crowdfunding založený na darech - každá kampaň, ve které není finanční návratnost pro 
investory
\item Crowdfunding založený na odměnách - kampaň, která zahrnuje jednotlivce přispívající 
na produkt výměnou za odměnu(výrobek nebo služba)
\item Crowdfunding zaležený na vlastním kapitálu - kampaň, která umožňuje přispěvatelům aby 
se stali součástí vlastníků společnosti
\end{itemize}

%popis navrhu GIS
\kapitola{Návrh informačního systému}
Před započetím samotné implementace by se měl sestavit návrh informačního systému, pro geografický 
informační systém to platí také.

\sekce{Případy užití}
Use Case model je jedním z UML modelů. V překladu lze diagram označit jako diagram případů užití. 
Jak již z názvu vyplývá, jedná se o popis chování systému z pohledů jeho uživatelů, proto je 
vytvářen mezi prvními diagramy. Model samotný se skládá z několika entit, mezi které patří 
actors(aktéři),  use case(případy užití) a vztahy mezi nimi. Definují se zde interakce mezi rolemi 
a systémem, kde role nemusí být pouze uživatel jako člověk, ale také jiný systém.

Jednotlivé připady užití mohou být složeny z více akcí vedoucích k dosažení požadovaného cíle. 
Případ užití je vykreslován jako elipsa s názvem. Aktéři jsou zobrazeni jako jednoduché postavy, 
které komunikují s případy užití. 

\podsekce{Diagram případů užití}
V diagramu jako jednotlivý aktéři operují běžní uživatelé, přihlášení uživatelé a administrátoři, 
kteří s daným systémem komunikují. Příhlášení uživatelé a administrátoři se do systému musejí 
nejprve přihlásit aby mohli dle svým pravomocí provádět dané činnosti. Běžní uživatelé pak mají 
zpřístupnění základní funkcionalitu, kterou jako návštěvníci webové aplikace očekávají. Nepřihlášení 
uživatelé mají možnost vyhledávání uložených informací, exportování seznamů rostlin, tisk mapy 
a práce s mapou jako takovou, ve smyslu posouvání aktuálního zobrazení mapy, přibližování, změny 
vrstvy, zobrazení detailních informací jednotlivých záznamů. Přihlášení uživatelé mají možnost 
spravovat svoje uložená data a přidávat nové. Administrátoři májí možnosti stejné jako přihlášení 
uživatelé, ke kterým mají možnost uživatele spravovat. 
%*******obrázek use case*********
\obrazek
\vlozeps{use_case.png}{0.45}
\endobr{Use Case.}

\newpage
\sekce{Mind mapa}
Mind map neboli myšlenková mapa je grafický návrh klíčových akcí propojených vzájemnými vztahy. 
Používá se ve fázi plánování, přemýšlení řešení problému, v této práci jak bude uživateli dovoleno 
pracovat v systému. V grafu se začíná vždy hlavní myšlenkou, Zobrazení mapy, která vystihuje hlavní
zobrazení aplikace. Kolem středobodu jsou tvořeny myšlenky vycházející z prvotní. Druhou větší 
myšlenkou lze označit Seznam objektů zastávájící podobnou funci a význam, kdy se jedná o tabulkové 
zobrazení všech uložených objektů. Seznam objektů je uveden celkem dvakrát, jednou v hlavní mapě 
určené pro běžného uživatele aplikace, podruhé je hlavním myšlenkou po přihlášení autorizovaného 
uživatele.
%*******obrázek use case*********
\obrazek
\vlozeps{mind_map.png}{0.45}
\endobr{Mind map.} 

\newpage
\sekce{Grafický návrh}
Všechny druhy aplikací se neobejdou bez dobře navrženého rozhraní, vzhled prodává a je prvním co 
uživatele zaujme. Aktuální webové stránky by měla být univerzální, z pohledu možnosti zobrazení, 
protože existuje velké množství přístrojů, které lze použít k jejich prohlížení, mezi které patří 
například chytrý mobilní telefon, tablet, stolní počítač. Obor zabývající se grafickým návrhem lze 
označit jako webdesign, zahrnující různé disciplíny a obory z oblasti tvorby webových stránek. 
Obvykle se termín webdesign používá k označení popisu procesu tvorby front-end části aplikace, 
protože z části zasahuje do webového inženýrství v šurším rámci vývoje. Historicky se webové stránky
vyskytovaly nejdříve pouze ve formě textu, tehdejší prohlížeče neměli integrovaný přístup k obrázkům
a jiným grafickým prvkům. V rámci zlepšování hardwaru a všudepřítomných konkurenčních bojů došlo 
k vývoji nových technogolií, zejména kaskádových stylů(CSS), nových možností pomocí značkovacího 
jazyka HTML, nebo JavaScriptu. Rozšíření internetu a jeho každodenní používání mělo za následky 
vytvoření nových standartů pro HTML(HTML5) a CSS(CSS3), které přinesly zcela nové možnosti nebo
nahrazení a vylepšení již existujících.
GIS arboreta jakožto webové stránky je typem aplikace s jedním oknem měnícím svůj obsah. Uvedený typ
je v dnešná době obvyklý a uživatelé jsou na něj zvyklí. Nejvíce používaným zařízením pro zobrazení 
aplikace bude s největší pravděpodobností chytrý mobilní telefon, nebo tablet. Běžní uživatelé jej 
využijí k prohlížení informací přimo u rostliny, u které se nacházejí. Stejně tak bude možno 
i zadávat data, přesně zaměřená dle aktuálního místa. Vzor One-Window Drilldown, řešící strukturu, 
vychází ze zobrazení v jednom okně, které přesně odpovídá požadavkům. Vzor zobrazuje aplikaci 
v celém okně pokaždé i po vykonání určité činnosti uživatele.
\sekce{Návrh jednotlivých sekcí}
Prvním návrhem je úvodní obrazovka, která obsahuje především zobrazení mapy. Manipulace s mapou 
vyžaduje alespoň základní ikony pro ovládání vzdálenosti, aktuální pozice a možnost změnit 
podkladovou vrstvu. Přihlášení uživatelé mají možnost pomocí tlačítka se symbolem plus přidávat nové 
body. Všechny ovládací prvky jsou situovány do pravé strany obrazovky, protože nejvíce přístupů bude 
pravděpodobně z mobilních telefonů a větší část jsou praváci, tedy používající zejména pravou ruku.
Dalším prvkem je umístění ikony pro zobrazení menu v levém horním rohu označeném třemi vodorovnými
rovnoběžnými čárkami.

\obrazek
\vlozeps{graficky_navrh_1.png}{0.30}
\endobr{Úvodní obrazovka.}

Menu bude obsahovat položky pro přihlášení, písemný seznam rostlin atd. Rozbalením se položky 
zobrazí a následným kliknutím na ikonu se zabalí. Pravý panel k ovládání mapy bude možno skrýt. 
Skrtí se provede automaticky po pohybu v mapě a zobrazí se tlačítko pro jeho opětovné zobrazení. 
Dalším důležitým prvkem návrhu je zobrazení rychlého detailu vybraného bodu. Po kliknutí na objekt 
bude objekt zvýrazněn a také se v levém spodním rohu objeví čtverec s rychlým náhledem detailu 
a možností prokliku do celkového detailu mimo zobrazení mapy.

\obrazek
\vlozeps{graficky_navrh_2.png}{0.30}
\endobr{Úvodní obrazovka.}

\obrazek
\vlozeps{graficky_navrh_3.png}{0.30}
\endobr{Úvodní obrazovka.}



%popis samotneho navrhu databaze
\kapitola{Návrh databáze}
Kompletní aplikace by měla být postavena na kvalitním databázovém návrhu. Návrh vychází 
z bakalářské práce Geodatabáze pro GIS arboreta MENDELU. Databáze je hlavním základním 
kamenem a bez její správné struktury by aplikace mohla být například pomalá nebo špatně udržitelná 
vzhledem k uložení obsahu. Návrh, z kterého bylo vycházeno, obsahoval nedostatky, které bylo 
nutné eliminovat. Prvním špatným aspektem bylo složité rozšíření jazykových mutací aplikace. 
Původně se zamýšleli pouze dvě mutace(česká a slovenská) s uchováním latinského názvu roslin 
a jejich hierarchické struktury. 

Moderní aplikace musí splňovat jazykovou rozšiřitelnost, proto 
bylo nutné návrh upravit pro uvedené potřeby. Existují tři populární způsoby návrhu databáze 
pro vícejazyčné webové stránky. Nejjednodušším řešením je již v původním návrhu použitá metoda sloupce, 
kdy každý jazyk je uchován v jednom sloupci tabulky. Hlavními klady jsou jednoduchá implementace, 
snadné provádění dotazů a neduplicitní obsah v rámci řádků(duplicitní data mohou vznikat 
v sloupcích, tedy jednotlivých jazycích). Jak bylo řečeno dříve je metoda špatně rozšiřitelná 
o nové jazykové mutace z pohledu zásahu do již fungující databáze a jejích dat, jelikož uložených
informací může být velké množství. Další možností je podobný způsob jako předchozí, ovšem s opačným 
provedením. Místo uložení jazyků do sloupců se uchovají jazyky v řádcích, tedy původně jeden záznam 
bude mít více záznamů. Poslední metoda, která je použita v návrhu, používá doplňkovou tabulku, která 
umožňuje shromáždit názvy v definovaných jazycích. Kladné vlastnosti převažují záporné, nejvíce lze
vidět jednoduchou jazykovou rozšiřitelnost(nejsou nutné úpravy schématu) a jednoduché složení dotazů.
Naopak hlavním negativem je možné zdojnásobení původních tabulek, protože se musejí vytvořit nové 
pro každou tabulku s požadovaným přeloženým názvem. Samotné názvy tedy nejsou uchovávány přímo 
v tabulkách rostlina, třída, atd., ale ve vazebních tabulkách spojující požadovanou tabulku
(roslina, třída, atd.) s tabulkou jazyk. Tabulka jazyk umožňuje uchovat jazyky, s kterými aplikace 
bude pracovat. Po provedení výše popsaných změn je již jednodušší rozšířit aplikaci o další jazyk, 
samozřejmě za předpokladu doplnění názvů všech uložených dat. Bylo nutné navrhnout pomocné spojovací 
tabulky s tabulkou jazyk, do které se přesunul sloupec \texttt{nazev} z požadovaných tabulek. 
Dalšími změnami jsou nové sloupce \texttt{datum\_vlozeni} a \texttt{datum\_upraveni} pro přehlednost 
manipulace s daty.

%*******obrázek databáze pro objekty*********
\obrazek
\vlozeps{navrh_db_blank_horni.png}{0.30}
\endobr{Část databáze pro objekty.}
\newpage

%http://www.ibot.cas.cz/botanika/o-taxonomii-a-systematice.html
Další problémovou částí návhru byla hierarchie zařazení roslin. Biologická klasifikace rostlin je 
způsob, jakým se kategorizují rostliny. Věda zabávající se právě tříděním organismů je označována 
jako taxonometrie, která kromě třídění zkoumá i vzájemné přibuznosti a podobnosti. V současnosti je 
používán hierarchický systém založený na Linneově klasifikačním systému, který vytvořil švedský 
biolog Carl Linné, považovaný za otce moderní biologie. Systém klasifikace, používaný v současné 
době, je doplněn o nové poznatky a je univerzální pro všechny organismy. Jednotlivé organismy 
vzájemně ražené do různých úrovní(říše, třída, řád, atd.) se nazývají taxony, kdy vyšší taxony 
mohou obsahovat jeden, nebo více taxonů nižší úrovně. Jednotlivé taxonomické jednotky se nazývají:
\newpage
\begin{itemize}
\item říše
\item podříše
\item oddělení
\item třída
\item podtřída
\item řád
\item podřád
\item nadčeleď
\item čeleď
\item podčeleď
\item rod
\item podrod
\item druh
\item poddruh
\item odrůda
\end{itemize}

%*******obrázek databáze pro strukturu*********
\obrazek
\vlozeps{navrh_db_blank_spodni.png}{0.28}
\endobr{Část databáze pro hierarchii.}
\newpage

Zásadní novou částí je oddělení administrace, proto bylo nutné navrhnout rozložení databáze pro 
tento účel. Hlávními aktéry systému jsou příhlášení uživatelé, pro které je určena tabulka 
\texttt{uzivatel}. Samotná tabulka k ovládání prostředí aplikace nestačí, uživatelé mohou mít 
rozlišné pravomoce ve smyslu, že někteří mohou přidávat, upravovat a mazat položky, jiní mohou 
pouze přidávat nové. Pravomoce jsou uchovány tabulkou skupina, kdy každý uživatel může mít jednu 
nebo více skupin.

%*******obrázek databáze pro administraci*********
\obrazek
\vlozeps{navrh_db_blank_admin.png}{0.28}
\endobr{Část databáze pro administraci.}


\kapitola{Praktická část}
\sekce{Modulový systém - OPP}
\sekce{Zabezpečení}
\sekce{Administrace}
\sekce{Mapa}
\sekce{Používání}
	\podsekce{Funkce pro GPS - mobil}
	\podsekce{Zadávání dat}
	\podsekce{Řešené problémy}

\kapitola{Závěr}


\begin{literatura}

\citace{designing-geodatabases}{Arctur, 2004}{
	\autor{ARCTUR, D. -- ZEILER, M} 
	\nazev{Designing geodatabases: case studies in GISdata modeling.}
	 Redlands: ESRI Press, 2004. 393 s. ISBN 1-58948-021-X}

\citace{intro-gis}{}{
	\autor{BERNHARDSEN, Tor} 
	\nazev{Geographic information systems: an introduction.}
	 3rd ed. New York: John Wiley, c2002, xiii, 428 s. ISBN 04-714-1968-0}

\citace{boundless}{Boundless, 2013}{
	\autor{BOUNDLESS} 
	\nazev{Boundless : Spatial Database Tips and Tricks : Introduction.}
	 Boundless, formerly OpenGeo [online]. 2013 [cit. 2014-05-02]. Dostupné z: http://workshops.boundlessgeo.com/postgis-spatialdbtips/introduction.html}

\end{literatura}

\prilohy

\end{document}
