\pdfoutput=1
\documentclass[12pt]{article}%
\usepackage[utf8]{inputenc} % pouľito v případě jiného kódování
% aktuální kódování: ISO Latin 2 (ISO 8859-2)
\usepackage{czech, dipp8}
\usepackage[usenames,dvipsnames]{color}
%\usepackage[usenames,dvipsnames,svgnames,table]{xcolor}

\begin{document}

%brava pro pozadi boxu urceneho pro prikazy
\definecolor{light-gray}{gray}{0.95}

\def\,{\penalty10000\hskip.25em}
\pagestyle{headings}

\cislovat{2}
\bakalarska

\titul{Tvorba geografického informačního systému arboreta MENDELU}
{Bc. David Ruber}{Ing. David Procházka, Ph.D.}{Brno 2014}

\newpage
Zde prostor pro zadání práce
\newpage

\podekovani{Rád bych poděkoval svému vedoucímu práce Davidu Procházkovi za ochotu, odborné
konzultace v rámci této práce.}

\prohlasenimuz{V~Brně dne \today}

\abstract{Bc. Ruber, D. Creating a geographic information system of MENDELU arboretum.
Diploma thesis. Brno: MENDELU, 2016}
{abstrakt}
\\
\\
{{\noindent{\sffamily\bfseries Keywords}}
\\
{klicova slova}

\abstrakt{Bc. Ruber, D. Tvorba geografického informačního systému arboreta MENDELU. Diplomová práce.
Brno: MENDELU, 2016}
{abstrakt}
\\
\\
{{\noindent{\sffamily\bfseries Klíčová slova}}
\\
{klicova slova}

\cislovat{3}
\obsah


\kapitola{Úvod a~cíl práce}
\sekce{Úvod}


\sekce{Cíl práce}

%predstaveni dostupneho sw
\kapitola{Dostupný software}
%popis djanga
%https://www.djangoproject.com/
%http://www.djangoproject.cz/
\sekce{Django}
Django je webový framework napsaný v~jazyce Python. Byl vyvinut aby pomohl vývojářům urychlit 
vývoj aplikací od~základního konceptu až po finální dokončení. K~těmto účelům poslouží jak oficiální 
návod krok po~kroku, ve~kterém jsou popsány všechny základní kroky, tak i~volně dostupná 
dokumentace obsahující všechny specifika. Django obsahuje velké množství již kompletně připravených 
modulů, které lze využít, jako jsou například autentizace uživatelů, mapu stránek, RSS kanál a~další. 
Ověřování uživatelů je jedním z~bezpečnostních prvků, které usnadňují práci vývojářům. Mezi další
bezpečností prvky patří SQL injection\footnote{SQL injection je technika napadení databázové vrstvy 
prostřednictvím neošetřeného vstupu.}, cross-site scripting\footnote{Metoda využívající bezpečnostní 
chyby ve skriptech a~vložení vlastních skriptů.}, cross-site request forgery\footnote{Metoda 
pracující na~bázi nezamýšleného požadavku pro~vykonání určité akce.} a clickjacking\footnote{Způsob 
útoku kdy uživatel spustí akci, kterou nepředpokládal.}. Dajngo využívá architekturu "sdílené nic", 
což znamená že~každý uzel je soběstačný a~nezávislý. Přesněji žádný z~uzlů nesdílí pamět nebo 
diskové úložiště. Framework odděluje jednotlivé komponenty (datábazovou vrstvu, aplikační vrstvu 
atd.).
%popis geo djanga
%https://docs.djangoproject.com/el/1.10/ref/contrib/gis/
\podsekce{GeoDjango}
GeoDjango je samostatný modul pro Django, který obsahuje geografické rozšíření pro~webové aplikace. 
Modul usiluje o~to, aby byl, stejně jako samostatný Django framework, co nejvíce jednochý pro 
vývojáře. Obsahuje vlastní modelá pole pro OGC\footnote{Open Geospatial Consortium je mezinárodní 
standartizační organizace.} geometrii a~rastrová data. Rozšiřuje Django ORM\footnote{Objektově 
relační zobrazení je programovací technika zajišťující konverzi dat mezi relační databází 
a~objektově orientovaným pragramovacím jazykem} pro potřeby geografických požadavků a~prostorových 
dat. Python rozhraní pro geometrii, rastrové operace a~práci s~daty v různých formátech je na 
vysoké úrovni.

%popis google maps
%https://www.google.com/maps/about/
%https://developers.google.com/maps/documentation/api-picker
\sekce{Google Maps}
Nejznámější mapovou službou jsou bezpochyby Google Maps. První verze Google Maps API vzinkla 
v roce 2005. API není jen JavaScript pro koncového uživatele, zahrnuje více projektů. Nabízí 
několik různých způsobů jak službu vložit do webových stránek s hodně možnostmi přizpůsobení. 
Dostupné API služby zahrnují například Google Maps API Javascript, Google Static Maps API, Google 
Maps SDK pro iOS nebo Google Maps pro Android. Základní APi jsou zdrama, ale platí určitá omezení. 
Prvním je, že mapy musí být volně a veřejně přístupné. Druhým je omezení využití, které omezuje to, 
aby web negeneroval více než povolené denní množství map. Existuje i placené řešení API.

%popis leafletu
%http://leafletjs.com/
\sekce{Leaflet}
Leaflet je nejrozšířenější open-source knihovna v~jazyce JavaScript pro tvorbu interaktivních map. 
Obsahuje většinu potřebných mapových funkcí, jako například:
\begin{itemize}
\item podkladové mapy, WMS\footnote{Web Map Service je otevřený standart pro~sdílení geografických 
informací.}
\item vektorové vrstvy: polygony, linie atd.
\item mapové značky
\item GeoJSON
\end{itemize}
Knihovna obsahuje již naimplementované prvky pro~ovládání mapy, mezi které patří gesta prsty nebo 
dvojité kliknutí myši pro příblížení, ovládání skrze klávesnici (pohyb pomocí šipek), nebo 
připravené ovládací panely zobrazené přímo v~mapě. Leaflet je navržen tak, aby byl jednoduchý, 
výkonný a~použitelný. Jeho hlavním kladem je funkčnost skrze všechny internetové prohlížeče 
stolních počítačů i mobilních zařízení. Samotné jádro knihovny klade důraz na jednoduchost, 
výkonnost a použitelnost.

%popis openlayers
%https://openlayers.org/
\sekce{OpenLayers}
OpenLayers je stejně jako Leaflet jedna z~nejrozšířenějších open-source JavaScript knihoven pro 
dynamické mapy umístěné na webových stránkách. Byl vyvinut společností MetaCraft jako ekvivalent 
ke~Google Maps. Umožňuje zobrazovat podkladové mapy, vektorová data, mapové značky a~další. 
Podkladové mapy je možnénačíst z mnoha zdrojů, např: OSM, Bing, MapBox. Stejně tak jsou 
podporovány OGC mapové služby. OpenLayers umožňuje i~vykreslení vektorových vrstev GeoJSON, KML, 
GML a~dalších formátů.

\sekce{Srovnání Leaflet a OpenLayers}
Porovnat Leaflet s OpenLayers je v celku jednoduché. Leaflet je zaměřen na jednoduchost a proto 
jeho použití je jednodušší. Je správným řešením pro jednoduché bežné mapové úkoly jako je zobrazení 
mapových podkladů a částí, posouvání mapy nebo zoomování. Celkové API je na vyšší úrovní, což 
znamená méně kódu a je jednodušší na pochopení, není tak rozsáhlé jako u OpenLayers. Openlayers 
nabízí více funkcí ovšem za cenu většího množství kódu, nutnosti inicializace a nastavení mapy. 
API je na nižší úrovní a proto je težší na pochopení a naučení se potřeb pro použití.
Díky inicializaci je dosáhnuta vyšší kontrola nad mapou a daty. V rámci kontroly je dobře zvládnuta 
podpora editace. Leaflet nabízí méně funkcí, ale existuje mnoho plug-in\footnote{Plug-in je 
zásuvný modul, který doplňuje aplikaci.}, které chybějící prvky doplní.

%popis crowdfundingu
%https://www.fundable.com/learn/resources/guides/crowdfunding-guide/what-is-crowdfunding
\kapitola{Crowdfunding}
Crowdfunding je způsob získávání pěnez na~projekt formou veřejné sbírky. Nejedná se o~tradiční 
přístup k~financování podnikatelské činnosti. Většinou je nutné provést průzkum trhu, nastavit 
svůj podnikatelský plán vytvořit prototyp a~následně produkt propagovat před investory. Zdroje 
financování jsou pak banky, investoři, nebo společnosti poskytující kapitál. Ve výsledku podobným 
přístupem je na konci omezené množství investorů. Crowdfunding tvoří přesný opak, na konci lze mít 
velké množství investorů. Pro získání peněz postačí vytvořit profil produktu na jedné z~mnoho 
webových stránek, nabídnout za příspěvek například první vyrobený kus, oficiální uvedení 
přispěvatele nebo jiné bonusy a~potom jen čekat na příspěvky, dokud není naplněn základní kapitál.
\sekce{Výhody crowdfundingu}
Existuje mnoho výhod oproti tradičním metodám, uvedené jsou jen některé z nich. Při použití 
crowdfunding platformy je získán přístup k tisícům akreditovaných investorů, kteří mohou 
komunikovat a dále předávat prezentovaný produkt. Vytvoření prezentace pomůže mapování historie 
vývoje produktu, jeho statistik z pohledu úspěšnosti a hlavně jako zpětná vazba od obyčejných 
lidí(investorů), kteří mohou vyjádřti svůj názor a tím přispět k novým vylepšením. Přístup 
získáva na~popularitě zejména díky sociálním sítím, na kterých se projekty mohou snadno šířit 
a díky tomu získávat více potencionálních přispěvatelů. Kromě sociálních sítí lze projekt šířit 
pomocí emailových newsletterů nebo vlastní webovou prezentací. 
Stejně jako klasických metod pro zahájení výroby, tak i typů crowdfundingu existuje více. Mezi 
základní patří:
\begin{itemize}
\item Crowdfunding založení na darech - každá kampaň, ve které není finanční návratnost pro 
investory
\item Crowdfunding založený na odměnách - kampaň, která zahrnuje jednotlivce přispívající 
na produkt výměnou za odměnu(výrobek nebo služba)
\item Crowdfunding zaležený na vlastním kapitálu - kampaň, která umožňuje přispěvatelům aby 
se stali součástí vlastníků společnosti
\end{itemize}

%popis samotneho navrhu databaze
\kapitola{Návrh databáze}
Před samotnou tvorbou databáze je potřeba rozhodnout jaká data se budou zaznamenávat. Hlavní požadavek
je, že se budou evidovat jednotlivé rostliny nacházející se v arboretu. Záznamy by měly obsahovat důkladný 
popis rostlin. Dostupné informace o rostlině, které se nacházejí na informační tabulce přímo u jednotlivých
exemplářů, ukazují nutnost zaznamenání latinského názvu, třídy, čeledi, řádu, identifikačního čísla 
přiřazeného jednotlivým rostlinám, poznámky (například krátký popis). Dalším požadavkem je možnost 
evidence cest, budov, informačních tabulí a~dalšího vybavení, které je součástí areálu. 
Při návrhu databáze byl využit grafický modelovací nástroj pgModeler - PostgreSQL Database Modeler. 
Databáze se skladá z~16 tabulek.
\sekce{Popis tabulek}

%*******Návrh databáze rostliny*********

Z prvního požadavku je vytvořena tabulka \texttt{rostlina} :
\paragraph{rostlina}
\begin{itemize}
\item Sloupce: gid\_rostlina (serial), id\_namereny\_bod (integer), nazev\_cz (varchar),
nazev\_lat (varchar), nazev\_sk (varchar), id\_trida (integer), id\_celed (integer), id\_rad (integer),
poznamka (text), identifikacni\_cislo (varchar), koruna (point)
\item Primární klíč: gid\_rostlina
\item Popis: Přiřazené informace k naměřenému bodu. Názvy rostliny se mohou vyskytnout v české, latinském 
a~slovenském jazyce. Rostliny mají přiřazenou třídu, čeleď a řád, které jsou řešeny pomocí cizích klíčů.
Sloupec \texttt{identifikacni\_cislo} je určen pro uložení čísla označujícího každou rostlinu. U~stromů, keřů a polokeřů
je možné evidovat i šířku koruny, tento sloupec je nepovinný.
\end{itemize}

%*******obrázek databáze pro rostlinu*********
\obrazek
\vlozeps{arboretum3_schema_spodni.png}{0.45}
\endobr{Část databáze pro rostlinu.}

Číselníky pro všechny třídy, řády a čeledě:
\paragraph{trida}
\begin{itemize}
\item Sloupce: id\_trida (serial), nazev\_cz (varchar), nazev\_lat (varchar), nazev\_sk (varchar)
\item Primární klíč: id\_trida
\item Popis: Přehled tříd s možností uložení ve 3 jazycích.
\end{itemize}

\newpage
\paragraph{celed}
\begin{itemize}
\item Sloupce: id\_celed (serial), nazev\_cz (varchar), nazev\_lat (varchar), nazev\_sk (varchar)
\item Primární klíč: id\_celed
\item Popis: Přehled čeledí s možností uložení ve 3 jazycích.
\end{itemize}

\paragraph{rad}
\begin{itemize}
\item Sloupce: id\_rad (serial), nazev\_cz (varchar), nazev\_lat (varchar), nazev\_sk (varchar) 
\item Primární klíč: id\_rad
\item Popis: Přehled řádů s možností uložení ve 3 jazycích.
\end{itemize}

\newpage
Tabulka \texttt{namereny\_bod} je určena pro uložení naměřených rostlin. Lze ji také vyžít jako vrstvu 
(shapefile) pro měření v terénu.
\paragraph{namereny\_bod}
\begin{itemize}
\item Sloupce: gid\_namereny\_bod (serial), nazev (varchar), geom (point)
\item Primární klíč: gid\_namereny\_bod
\item Popis: Sloupec \texttt{nazev} lze považovat za pracovní označení rostliny při měření v terénu
\end{itemize}

%*******Návrh databáze zbytek*********
Návrh části databáze pro rostliny byl představen, nyní je potřeba představit zbylou část. Hlavní pozornost
je zde věnována tabulce \texttt{objekt}, která shrnuje všechny objekty. 
\paragraph{objekt}
\begin{itemize}
\item Sloupce: id\_objekt (serial), objekt (integer)
\item Primární klíč: id\_objekt
\item Popis: Hlavní účel tabulky slouží ke spojení fotek s objektem (budova, rostlina, ...). 
Sloupec \texttt{objekt} je cizí klíč odkazující na tabulky: \texttt{znacka}, \texttt{cesta}, 
\texttt{budova}, \texttt{vybaveni}. 
\end{itemize}

Uložení je cest je provedeno rovnou ve 3 tabulkách \texttt{cesta}, \texttt{cesta\_typ\_cesty} 
a~\texttt{typ\_cesty}. Cesty mohou mít přiřazené typy, které ji definují (například: pěší, schody, průjezdná atd.)
a umožňují ji ve výsledné mapě rozlišovat.
\paragraph{cesta}
\begin{itemize}
\item Sloupce: gid\_cesta (serial), nazev (varchar), geom (multilinestring), popis (text)
\item Primární klíč: id\_cesta
\item Popis: Uložené cesty arboreta s názvem a popisem.
\end{itemize}

\paragraph{cesta\_typ\_cesty}
\begin{itemize}
\item Sloupce: id\_cesta\_typ\_cesty (serial), id\_cesta (integer), id\_typ\_cesty (integer), 
nazev (varchar)
\item Primární klíč: id\_cesta\_typ\_cesty
\item Popis: Tabulka reprezentuje vztah M:N pro \texttt{cesta} a \texttt{typ\_cesty}. 
Sloupec nazev není povinný a je určen k jednoznačnému rozpoznání části cesty.
\end{itemize}

\paragraph{typ\_cesty}
\begin{itemize}
\item Sloupce: id\_typ\_cesty (serial), nazev (varchar), oznaceni (varchar), popis (text)
\item Primární klíč: id\_typ\_cesty
\item Popis: Tabulka pro typy cest. Sloupec \texttt{oznaceni} je určen pro název nebo zkratku 
symbolu, kterým bude typ cesty označen v mapě.
\end{itemize}

Další důležitou součástí arboreta MENDELU jsou bezesporu informační tabule.
\paragraph{znacka}
\begin{itemize}
\item Sloupce: gid\_znacka (serial), nazev (varchar), hlavni\_cislo (integer), vedlejsi\_cislo(integer), 
popis (varchar), geom (point)
\item Primární klíč: gid\_znacka
\item Popis: Uložené informační tabule v areálu. Obsahují hlavní číslo oblasti a vedlejší číslo, které 
se nachází přímo na tabuly. Sloupec \texttt{poznamka} je určen pro případný popis oblasti.
\end{itemize}

Jeden z požadavků úvaděl potřebu evidence budov a ostatního vybavení arboreta.  Řešením jsou
tabulky \texttt{budova}, \texttt{vybaveni} a \texttt{typ\_vybaveni}. Vybavením jsou myšleny
všechny další důležité věci jako lavičky, vodovodní kohoutky, atd.
\paragraph{budova}
\begin{itemize}
\item Sloupce: gid\_budova (serial), nazev (varchar), popis (text), geom (multipolygon)
\item Primární klíč: gid\_budova
\item Popis: Tabulka pro budovy arboreta, které se ukládájí jako polygony.
\end{itemize}

\paragraph{vybaveni}
\begin{itemize}
\item Sloupce: gid\_vybaveni (serial), id\_typy\_vybaveni (integer), popis (text),
geom (point)
\item Primární klíč: gid\_vybaveni
\item Popis: Tabulka pro uložení ostatního vybavení arboreta.
\end{itemize}

\paragraph{typ\_vybaveni}
\begin{itemize}
\item Sloupce: id\_typ\_vybaveni (serial), nazev (varchar)
\item Primární klíč: id\_typ\_vybaveni
\item Popis: Přehled možných typů vybavení.
\end{itemize}

V dnešní době je téměr nemyslitelné zobrazovat data mapy bez reálné fotky určitého místa.
Typickým příkladem využití fotek s mapami mohou být Google mapy, které umožňují sledovat fotky
přiřazeném k místům nebo virtuální prohlídku místa, díky aplikaci Street View. Zde jsou fotky
přiřazovány k objektům arboreta vztahem M:N, takže jedna fotka může mít jeden nebo více objektů 
a~objekt může mít jednu nebo více fotek.

\paragraph{fotka}
\begin{itemize}
\item Sloupce: id\_fotka (serial), url (varchar), popis (text)
\item Primární klíč: id\_fotka
\item Popis: Sloupec \texttt{url} slouží k uložení url
\footnote{URL (Uniform Resource Locators) - Umístění webové stránky nebo souboru na internetu.} 
adresy na serveru, na kterém jsou nahrané fotky.
\end{itemize}

\paragraph{objekt\_fotka}
\begin{itemize}
\item Sloupce: id\_objekt\_fotka (serial), id\_objekt (integer), id\_fotka (integer)
\item Primární klíč: id\_objekt\_fotka
\item Popis: Pomocná tabulka pro vazbu M:N mezi tabulkami \texttt{objekt} a \texttt{fotka}.
\end{itemize}

%*******obrazek navrh databaze horni******
\obrazek
\vlozeps{arboretum3_schema_horni.png}{0.4}
\endobr{Návrh databáze.}
\newpage

\kapitola{Závěr}


\begin{literatura}

\citace{designing-geodatabases}{Arctur, 2004}{
	\autor{ARCTUR, D. -- ZEILER, M} 
	\nazev{Designing geodatabases: case studies in GISdata modeling.}
	 Redlands: ESRI Press, 2004. 393 s. ISBN 1-58948-021-X}

\citace{intro-gis}{}{
	\autor{BERNHARDSEN, Tor} 
	\nazev{Geographic information systems: an introduction.}
	 3rd ed. New York: John Wiley, c2002, xiii, 428 s. ISBN 04-714-1968-0}

\citace{boundless}{Boundless, 2013}{
	\autor{BOUNDLESS} 
	\nazev{Boundless : Spatial Database Tips and Tricks : Introduction.}
	 Boundless, formerly OpenGeo [online]. 2013 [cit. 2014-05-02]. Dostupné z: http://workshops.boundlessgeo.com/postgis-spatialdbtips/introduction.html}

\end{literatura}

\prilohy

\end{document}
